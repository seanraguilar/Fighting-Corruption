\documentclass{article}
\sloppy
\begin{document}
\title{Should Fighting Corruption Always Be One of the Main
Pre-Requisites for Economic Help?}
\author{Sean Aguilar and Vladik Kreinovich\\
Department of Computer Science\\
University of Texas at El Paso\\
500 W. University\\El Paso, TX 79968. USA\\sraguilar4@miners.utep.edu, vladik@utep.edu}
\date{}
\maketitle

\begin{abstract}
In general, corruption is bad. In many cases, it makes sense to make fighting corruption one of the main pre-requisites for getting financial help: we do not want this money to line the pockets of corrupted officials, we want to help the people. In this paper, we argue, however, that in some cases -- of over-regulated and/or oppressive regimes -- too much emphasis on fighting corruption may be counter-productive: instead of helping people, it may hurt them.
\end{abstract}

\section{Formulation of the Problem}

\noindent{\bf Fighting corruption is one of the main pre-requisites for economic help.} Many developing countries have a high level of corruption. Because of this, fighting corruption is usually one of the main pre-requisites for economic help.
\medskip

\noindent{\bf But is making this a pre-requisite always a good idea?} At first glance, the situation is very straightforward: corruption is bad, we should fight it; see, e.g., \cite{Jain 2001} and references therein.

However, what we plan to show is that there are nuances. The ultimate goal of providing economic help to a country is to help the people of this country. In many cases, requiring the recipient government to fight corruption more actively does help the people of this country, but not always -- there are cases when over-emphasizing this fight is detrimental to the economic prosperity -- and even to basic human rights. To clarify: we are not arguing that corruption is good, it is clearly bad, what we are arguing is that in some cases, there are other obstacles to people's happiness -- and in the presence of these obstacles, some level of corruption sometimes makes people's lives easier.
\medskip

\noindent{\bf What we plan to do in this paper.} In this paper, we first briefly recall the usual developed-countries negative attitude towards corruption. Then, we explain situations in which similar negative attitude makes sense for developing countries as well. And after that, we explain why sometimes, a certain level of corruption helps the population overcome other -- even more negative -- obstacles to their economic prosperity.

\section{Corruption in the Developed Countries: A Brief Reminder}

\noindent{\bf What is corruption: a brief reminder.} Corruption means two related things:
\begin{itemize}
\item when an official requires a bribe -- monetary or otherwise -- for activities that he/she is required to do for free, and
\item when an official, for a bribe, allows an action which is prohibited by the country's laws and regulations.
\end{itemize}
\medskip

\noindent{\bf Let us give simple economic examples.} Suppose that a person wants to open a restaurant to serve the local folks. Usually, to open a restaurant, one needs to make sure that the corresponding sanitary, fire, etc., regulations are satisfied. These regulations are checked by the corresponding inspectors.

An example of the first type of corruption is when an inspector, while inspecting a perfectly suitable establishment, does not certify it right away but instead solicits a bribe. From the economic viewpoint, this is a clearly parasitic waste of resources: the inspector does not produce any social good, instead, he/she parasitically takes money from the restaurant owner -- who will then have to compensate for this by charging customers more. No one benefits except for the inspector. Everyone else suffers: the owner loses money, the customers lose money. This is a bad situation, there is nothing good about it.

An example of the second type of corruption is when an official, when selecting a food service to serve an official event, selects not the best and not the cheapest (for the same quality) service: instead, he/she selects an inferior service because the owner of that service paid him a bribe. In this case, everyone suffers: the service is inferior, the costs are higher -- and these costs come from people's taxes. This is also a bad situation. there is nothing good about it.

An even worse situation happens in the second type of corruption, when the inspector, for a bribe, certifies an unhealthy (or fire-hazardous) restaurant as supposedly ready to serve customers. In this case, not only money is lost: customers may get sick from unhealthily prepared food -- or even die in a possible fire. This is a very bad situation, there is nothing good about it.
\medskip

\noindent{\bf Resulting attitude to corruption.} Unfortunately, sometimes these situation occur. Everyone understands that these are bad situations, so the general attitude to such situations is negative. Every time there is corruption, people suffer.

Not surprisingly, when a developing country asks for financial help, and it is known that this country has a significant level of corruption, a natural idea is to make fighting corruption one of the main pre-requisites for providing this help.

\section{In Many Situations, Requiring Corruption-Fighting Measures Makes Perfect Sense}

\noindent{\bf What type of situation do we consider.} In many countries, laws and regulations are reasonable -- at least on the paper. Legally speaking, e.g., a person who opens a restaurant can do it if he/she follows reasonable rules.

In practice, however, this is not so easy, because this requires several official approvals by different agencies, and a significant proportion of the corresponding officials solicits bribes.
\medskip

\noindent{\bf In such situations, the corruption-fighting requirements are well-justified.} In such situations, corruption causes the same problems as in the developed countries -- they are pure evil.

A country or organization that provides financial help does not want a portion of its money -- which is intended the help the people -- to simply line the pockets of corrupt officials. And such lining-the-pockets did occur in the past with financial help.

To make sure that this help helps the people, a reasonable idea is to require the government to fight the existing corruption as much as possible.

\section{But There Are Other Situations As Well}

\noindent{\bf Possible situations.} In many cases, the laws and regulations are excessive: there are so many bureaucratic regulations that, if we obediently follow all of them, opening a restaurant would require a year -- or even longer.
\medskip

\noindent{\bf What happens in such situations.} In this case, the future owner has two options:
\begin{itemize}
\item he/she can obediently follow all the regulations and get his/her permission after a year or two of non-productive permission-seeking activity; or
\item he/she can pay a bribe -- often actively solicited by the bureaucrats -- and get the permission right away.
\end{itemize}
In the first option, during the one- or two-year interval, the owner loses money and the local folks lack the opportunity to enjoy a good meal.

In the second option, the local folks can enjoy the meals right away, and the owner can start earning money -- and thus paying back what he/she may have borrowed to start a business.
In this option, everyone of happy. Yes, this is not an ideal situation, money given to the corrupt official are wasted -- but even with this waste, the situation is better than the first option.
\medskip

\noindent{\bf How can we describe it in quantitative terms: general idea.} To describe the two options in quantitative terms, let us
consider them from the viewpoint of all the folks whose prosperity we care about:
\begin{itemize}
\item the owner and
\item the local folks who plan to frequent the restaurant.
\end{itemize}

Let us first consider the owner's viewpoint. Let us denote the amount of money that the owner spent to open the restaurant by $m$. In many cases, the owner need to borrow this money. Not only he/she borrows the amount $m$, he/she has to borrow more -- either to pay a bribe $b$ or to pay interest rate $r$ for the period when the restaurant is ready but lack an official permission to open. Once the restaurant opens, the owner will get annual profit $p$.

From the customers' viewpoint, once the restaurant opens, each of them can save an amount $a$ per year -- e.g., by saving on the public transportation that is needed if there is no good restaurant in the close vicinity of their workplaces.
\medskip

\noindent{\bf Quantitative description of the first option.}
In Option 1, the owner has to borrow the amount $A_0$ which is sufficient to cover both the original amount $m$ and the amount of interest $r\cdot A_0$ that he/she needs to pay the first year, when the restaurant still does not generate any profit. From the condition $A_0=m+r\cdot A_0$, we conclude that $A_0=\displaystyle\frac{m}{1-r}$. After the first year, the owner pays back the sum $r\cdot A_0$ and so, the remaining amount owed to the bank (or to whoever loaned him/her the money) will be $A_1=m$.

Once the restaurant opens, each year $t$, we start with the amount-owed $A_{t-1}$. From this amount, the owner can pay amount $p$. From this amount, the part $r\cdot A_{t-1}$ goes into paying interest, and the remaining part $p-r\cdot A_{t-1}$ goes towards the principal. So, at the end of the year, we get a smaller amount-owed $$A_t=A_{t-1}-(p-r\cdot A_{t-1})=(1+r)\cdot A_{t-1}-p.\eqno{(1)}$$

Of course, the profit should cover the interest -- otherwise, the restaurant will only lose money, so we must have $p\ge r\cdot m$.

There are known ways to solve this equation (see, e.g., \cite{Cormen 2009}). Namely, if we subtract $\displaystyle\frac{p}{r}$ from both sides of the equation (1), we conclude that $$A_t-\frac{p}{r}=(1+r)\cdot A_{t-1}-p-\frac{p}{r}=(1+r)\cdot \left(A_{t-1}-\frac{p}{r}\right).$$
Thus, the values $V_t\stackrel{\rm def}{=}A_t-\displaystyle\frac{p}{r}$ form a geometric progression, so $$V_t=V_1\cdot (1+r)^{t-1}=\frac{V_1}{1+r}\cdot (1+r)^t.$$

For $t=1$, we have $V_1=A_1-\displaystyle\frac{p}{r}=m-\displaystyle\frac{p}{r}$, thus $$V_t=\frac{m-\displaystyle\frac{p}{r}}{1+r}\cdot (1+r)^t=
\frac{r\cdot m-p}{r\cdot (1+r)}\cdot (1+r)^t.$$ Since $p>r\cdot m$, this value is negative. Thus, $$A_t=V_t+\frac{p}{r}=\frac{p}{r}-\frac{p-r\cdot m}{r\cdot (1+r)}\cdot (1+r)^t.$$ The owner will pay his/her debts in time $T$ for which $A_T=0$, i.e., for which $$(1+r)^T=\frac{p\cdot (1+r)}{p-r\cdot m}$$ and
$$T=\log_{1+r}\left(\frac{p\cdot (1+r)}{p-r\cdot m}\right).$$
\medskip

\noindent{\bf Quantitative description of the second option.}
In Option 2, the owner has to borrow the amount $m$ needed to open the restaurant and the amount $b$ of the bribe, so $A_0=m+b$. In this option, the owner starts getting profit right away, so from the formula (1), we conclude that $$V_t=V_0\cdot (1+r)^t=\left(m+b-\frac{p}{r}\right)\cdot (1+r)^t$$ and
$$A_t=\frac{p}{r}-\left(\frac{p}{r}-(m+b)\right)\cdot (1+r)^t=\frac{p}{r}-\frac{p-m\cdot r-b\cdot r}{r}\cdot (1+r)^t.$$ The owner will pay his/her debts in time $T$ for which $A_T=0$, i.e., for which $$(1+r)^T=\frac{p}{p-r\cdot m-r\cdot m}$$ and
$$T=\log_{1+r}\left(\frac{p}{p-r\cdot m-r\cdot b}\right).$$
\medskip

\noindent{\bf When is it beneficial for the owner to pay the bribe?} When it leads to a faster profitability, i.e., when $$\frac{p}{p-r\cdot m-r\cdot b}<\frac{p\cdot (1+r)}{p-r\cdot m}.$$ If we divide both sides by $p$ and multiply both sides by both denominators, we get
$$p-r\cdot m<(1+r)\cdot(p-r\cdot m)-(1+r)\cdot r\cdot b.$$ Moving the $b$-term to the left-hand side and the other terms to the right-hand side and simplifying the expression, we get $(1+r)\cdot r\cdot b<r\cdot (p-r\cdot m),$ i.e., equivalently, $$b<\frac{p-r\cdot m}{1+r}.$$ In other words, if the bribe is smaller than a certain amount, Option 1 is preferable.
\medskip

\noindent{\bf When is it beneficial for the local folks if the owner pays a bribe?} Always, since this way, they start saving the amount $a$ the first year.
\medskip

\noindent{\bf What will happen if the country starts fighting corruption.} Because fighting corruption is one of the main pre-requisites for getting financial help, the expected help-recipient country will actively fight corruption, and Option 1 will disappear.

Who benefits? No one.

\section{Beyond Economics}

\noindent{\bf Extreme case.}
Close to the end of the movie Schindler's List, when the Jews that Oscar Schindler tries to save are finally sent to the death camp, he bribes the camp's commander and saves their lives. Is this corruption? Absolutely. But this is one of the cases when corruption -- a necessary evil -- helps fight an ever greater evil.

Who would have benefited if the Nazi government eliminated corruption? Hitler would have benefited, since no Jews would be saved from destruction.
\medskip

\noindent{\bf Experience of one of the authors.} An experience of one of the authors -- VK -- living in an oppressive country (USSR)
showed that in an oppressive regime, corruption may be good. Let
me give you a few examples.

In the mid-1970s, he worked in Novosibirsk Akademgorodok,  an
academic campus 30 km from Novosibirsk, Russia, the capital of
Siberia.  There was no meat sold in the government stores (and
there were no other stores), as there was no meat sold in the
stores in most places in the country. When he arrived there in
1975, he was told that few years ago, meat and meat products were
available, but after several academicians from Akademgorodok
protested against the 1968 invasion of Czechoslovakia, the
previous special supply status was withdrawn.

People could buy meat at a farmer's market in Novosibisrk proper, but that
required a whole day: a long bus ride there and back; besides, the
price of everything on the market was huge -- since due to many
government restrictions, farmers could mostly only sell what they
grow locally; VK remembers an American researcher complaining that a
small watermelon cost the equivalent of \$15.

People could get some meat products in one of the local state-owned
cafeteria (and there were no others), but that was usually not
very tasty or nutritious. VK remembers a student newspaper
characterizing a meat sauce as a violet-colored nightmare (no
idea why it was violet-colored). So, people did a natural
thing: they would go to the back door of the cafeteria and offer to
buy (illegally) some meat and/or meat products.
Everyone was happy: folks who worked at the cafeteria got some money in their pockets on top of
their meager salaries, people got meat which they could prepare tastily at
home. Without this corruption, the people would be eating the
violet-colored nightmare.

A more serious issue. At some point during VK's stay in
Novosibirsk, he got seriously sick; as it turned out, he had a
dangerously low blood pressure. He got lucky: the doctor who came
with an ambulance was working on a PhD dissertation on exactly
this topic, so this doctor involved VK in his research experiments. After
the doctor finally diagnosed VK, he suggested a Swiss-produced medicine
that he knew to be perfect for VK's condition. This medicine was not
available in the state-owned drugstores (and there none else). So VK did
what everyone did in this situation: his Mom had a friend who was a
doctor, she knew a colleague who worked in a special KGB hospital
where they had access to all kinds of medicine, so VK got the
miracle medicine and was cured. It was all perfectly illegal,
with a ``gift" (bribe) going to a medical doctor for ``stealing"
this medicine. It was corruption. But without this
corruption, VK might have been seriously ill or even dead.

Another serious issue. At this time, many Jews wanted to emigrate
to the State of Israel. In Novosibirsk (and in St. Petersburg
where VK was born and lived until moving to Novosibirsk),
emigration was, at that time, not possible: you apply, you are
denied, and you (and all your family) are fired from your jobs
for being a potential traitor. However, many Jews managed to
emigrate: they moved first to Georgian Republic where it was known
that for a bribe, you can get a permission to leave. Corruption?
Yes, but without this corruption, thousands of people would not
have reached freedom.

The West was pressuring the Soviet government -- as it is
pressuring many oppressive governments now -- to be more vigilant
in fighting corruption. And the government gladly pushed harder.
In this, the government and the West pushed in the same direction.
The leadership of Georgian Republic was replaced, it became no
longer possible to emigrate. Several folks who took bribes in
other areas of the country went to jail, so it became not so easy
to get good food or to get good medicine. The corruption went
down. The Western economists were happy.

Were the people of the country happier? With less
opportunities to get a good food or a good medicine? With less
opportunities to leave the oppressive country? NO.

So please do not think that pushing oppressive regimes to fight
corruption always helps the common folks: in many cases, such a
fight only helps the oppressors.

\section*{Acknowledgments}

This work was supported in part by the National Science Foundation
grants 1623190 (A Model of Change for Preparing a New Generation
for Professional Practice in Computer Science), and HRD-1834620 and
HRD-2034030 (CAHSI Includes).

\begin{thebibliography}{9}

\bibitem{Cormen 2009} T. H. Cormen, C. E. Leiserson, R. L. Rivest, and C. Stein, {\it Introduction to Algorithms}, MIT Press,
Cambridge, Massachusetts, 2009.

\bibitem{Jain 2001} A. K. Jain, ``Corruption: a review'', {\it Journal of Economic Surveys}, 2001, Vol. 15, No. 1, pp. 71--120.

\end{thebibliography}
\end{document} 